%\section{Abstract}
Social networks are modeled as graphs with nodes representing individuals and edges denoting interactions between the individuals. 
A community is defined as a subgraph with dense internal connections and sparse external connections. Many heuristic algorithms have been proposed 
in the literature for disjoint community detection. Nodes that belong to more than one community form overlapping regions between communities. 
In this thesis, we proposed two novel overlapping community discovery algorithms based on the idea of consensus clustering. The first algorithm  
defines core members of  a community as  nodes that will be  co-clustered by all the algorithms as belonging to one community. The rest of the nodes,
unless on the periphery are potentially in the overlapping regions of communities. The second algorithm tries to retrieve overlapping nodes directly 
based on connectivity consensus. The intuition behind this algorithm is that the neighbours of an overlapping node most probably belong to  different 
communities. Here again the 'overlappingness' of a node is decided based on the consensus arrived at by majority of the community discovery algorithms. 
The two algorithms are implemented and tested on many benchmark data sets for community discovery. The algorithms successfully detected overlapping nodes 
that can be verified visually for small networks. Since overlapping node information is not available for benchmark networks, we designed two  additional 
data sets joining friendship networks of two friends using their individual facebook  data. It is shown that both the algorithms retrieve friends who have 
high degree of interactions with many other friends as overlapping nodes. These nodes are not merely those that belong to intersection of two communities 
but belong to common regions of many hidden communities that are discovered by these algorithms which could not have been inferred easily.

%As part of these projects we expect the students to develop some of the existing schemes and, if possible, to come up with new schemes realizing the compartmented access structures.
